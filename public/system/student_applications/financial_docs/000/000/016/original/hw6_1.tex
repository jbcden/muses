\documentclass[11pt]{article}
\usepackage{fullpage}
\usepackage{fancyhdr}
\usepackage{epsfig}
\usepackage{algorithm}
\usepackage[noend]{algorithmic}
\usepackage{amsmath,amssymb,amsthm}
\usepackage{enumerate}

\newtheorem{lemma}{Lemma}
\newtheorem*{lem}{Lemma}
\newtheorem{definition}{Definition}
\newtheorem{notation}{Notation}
\newtheorem*{claim}{Claim}
\newtheorem*{fclaim}{False Claim}
\newtheorem{observation}{Observation}
\newtheorem{conjecture}[lemma]{Conjecture}
\newtheorem{theorem}{Theorem}
\newtheorem{corollary}[lemma]{Corollary}
\newtheorem{proposition}[lemma]{Proposition}
\newtheorem*{rt}{Running Time}

\renewcommand{\dh}{\hat{\delta}}


%%% You can ignore the following stuff, it's just for formatting purposes
\textheight=8.6in
\setlength{\textwidth}{6.44in}
\addtolength{\headheight}{\baselineskip} 
% enumerate uses a), b), c), ...
\renewcommand{\labelenumi}{\alph{enumi})}
% Sets the style for fancy pages (namely, all but first page)
\pagestyle{fancy}
\fancyhf{}
\renewcommand{\headrulewidth}{0.0pt}
\renewcommand{\footrulewidth}{0.4pt}
% Changes style of plain pages (namely, the first page)
\fancypagestyle{plain}{
  \fancyhf{}
  \renewcommand\headrulewidth{0pt}
  \renewcommand\footrulewidth{0.4pt}
  \renewcommand{\headrule}{}
  }
% Changes the title box on the first page
\renewcommand\maketitle{
\begin{center}
\begin{tabular*}{6.44in}{l @{\extracolsep{\fill}}c r}
\hline
\bfseries  &  & \bfseries CS 383 Fall 2013 \\
\bfseries&  & \bfseries  Homework \# 5 Solutions\\ %%% UPDATE FOR EACH ASSIGNMENT
\bfseries   &   &  \bfseries Jacob Chae\\ %%% YOUR NAME HERE
\hline
\end{tabular*}
\end{center} }



\begin{document}
\maketitle
\thispagestyle{plain}


%%% PLEASE PLACE THE HONOR CODE AND YOUR NAME/SIGNATURE HERE
\noindent Honor Code: 

\noindent I affirm that I have adhered to the Honor Code on this assignment.\\\\
\textbf{Collaborators: Cooper Labinger, David Meyers}

\subsection*{Problem 1: Are we truly free?}

\textbf{Construction }\\
Let us create a PDA, M such that: \\
$$M=(Q, \Sigma, \Gamma, \delta, s, F, \perp)$$\\
M will have transitions defined as follows:\\
$$(q, a, A) \xrightarrow[M]{1} (q', \alpha)$$
$$(r, a, \perp) \xrightarrow[M]{1} (r', \perp)$$
Now let us create a second PDA, N that models a regular language:\\
$$N = (Q', \Sigma, \{\amalg\}, \delta ', s'. F', \amalg)$$
With transition:\\
$$(q, a, \amalg) \xrightarrow[N]{1} (\delta '(q, a), \amalg)$$\\
Finally, let us create a third PDA M' that models $L(M) \cap L(N)$\\
$$M'=(Q \times Q', \Sigma, \Gamma \times \{\amalg\}, \Delta, (s, s'), F \times F', (\perp, \amalg))$$\\
With transition:
$$((q, r), a, (A, \amalg)) \xrightarrow[M']{1} ((q', r'), (\alpha, \amalg))$$\\
As can be seen, all of our transitions are defined for some states q or r, some $a \in \Sigma$ and some $\alpha \in \Gamma$ that gets pushed onto the stack. \\
Let all of out PDA's define acceptance in terms of reaching some final state.

\begin{lemma}
$(q, xy, \alpha) \xrightarrow[M]{*} (q', y, \alpha ')$ and $(r, xy, \amalg) \xrightarrow[N]{*} (r', y, \amalg) \iff ((q, r), xy, \alpha \times \{\amalg\}) \xrightarrow[M']{*} ((q', r'), y, \alpha '\times \{\amalg\})$

\begin{proof}
$(\Rightarrow)$ Induction on the length of the derivation.\\
Base Case n = 0: $(q, xy, \alpha) \xrightarrow[M]{0} (q, xy, \alpha)$ and $(r, xy, \amalg) \xrightarrow[N]{0} (r, xy, \amalg) \Rightarrow ((q, r), xy, \alpha \times \{\amalg\}) \xrightarrow[M']{0} ((q, r), xy, \alpha \times \{\amalg\})$\\

\noindent Inductive Hypothesis: Assume this is true for n+1 derivations and let $a \in \Sigma$

\noindent Inductive Step:\\
$$(q, axy, \alpha) \xrightarrow[M]{1} (q, xy, \alpha) \xrightarrow[M]{n} (q, y, \alpha ')$$
$$(r, axy, \amalg) \xrightarrow[N]{1} (r, xy, \amalg) \xrightarrow[N]{n} (r, y, \amalg)$$
$$((q, r), axy, \alpha \times \{\amalg\}) \xrightarrow[M']{1} ((q, r), xy, \alpha \times \{\amalg\}) \xrightarrow[M']{n} ((q, r), y, \alpha \times \{\amalg\})$$\\

$(\Leftarrow)$ This direction is trivial as it is symmetrical to the opposite direction.
\end{proof}
\end{lemma}

\begin{theorem}

$$(q, a A) \xrightarrow[M]{1} (q', \alpha)$$ 
\center and 

$$(r, a, \amalg) \xrightarrow[N]{1} (\delta ' (r', a), \amalg)$$ 

\center are accept states \\
\center$\iff$
$$((q, r), a, \alpha \times \{\amalg\}) \xrightarrow[M']{1} ((q', r'), (\alpha, \amalg))$$ is also an accept state.

\begin{proof}
If M and N are in an equivalent state, then by our construction, M' is also in an equivalent state as proven by our lemma. Therefore, if M and N are in accept states, then M' must also be in an accept state.
\end{proof}
\end{theorem}









\end{document}

